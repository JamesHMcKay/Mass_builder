\documentclass[11pt]{article}
\usepackage{geometry}                % See geometry.pdf to learn the layout options. There are lots.
\geometry{letterpaper}                   % ... or a4paper or a5paper or ... 
%\geometry{landscape}                % Activate for for rotated page geometry
%\usepackage[parfill]{parskip}    % Activate to begin paragraphs with an empty line rather than an indent
\usepackage{graphicx}
\usepackage{amssymb}
\usepackage{slashed}
\usepackage{epstopdf}
\usepackage{amsmath}
\DeclareGraphicsRule{.tif}{png}{.png}{`convert #1 `dirname #1`/`basename #1 .tif`.png}

\setlength{\parindent}{0pt}

\usepackage{lmodern}
\usepackage{colortbl}

\usepackage{mmacells}
\usepackage{longtable}





\newcommand{\newc}{\newcommand}
\newcommand{\mychi}{\raisebox{0pt}[1ex][1ex]{\tiny$\chi$}}
\newcommand{\mychibig}{\raisebox{0pt}[1ex][1ex]{$\chi$}}

\newcommand{\chiinline}{\raisebox{1.7pt}[1ex][1ex]{$\chi$}}

\def\sp{\slashed{p}}
\def\sk{\slashed{k}}
\def\cn{\chi^0}
\def\cp{\chi^+}
\def\cm{\chi^-}
\def\gm{\gamma^{\mu}}
\def\gn{\gamma^{\nu}}
\def\gp{\gamma^{\rho}}
\def\km{k_{\mu}}
\def\kn{k_{\nu}}
\def\kp{k_{\rho}}
\renewcommand{\d}{\ensuremath{\operatorname{d}\!}}
\newc{\cTarasov}{a}


\def\cpp{\mychi^{++}}
\def\cmm{\mychi^{--}}
\def\Mp{M_{\text{pole}}}
\def\Mpa{M_{\text{pole},A}}
\def\Mpb{M_{\text{pole},B}}
\def\kmpm{k^{\mu}p_{\mu}}
\def\he{\frac{\epsilon}{2}}

\def\mc{m_{\mychi}}

\newcommand{\mb}{\textsf{Mass Builder} }
\newcommand{\mbs}{\textsf{Mass Builder}}
\newcommand{\tsil}{\textsf{TSIL} }
\newcommand{\tsils}{\textsf{TSIL}}
\newcommand{\tarcer}{\textsf{TARCER} }
\newcommand{\tarcers}{\textsf{TARCER}}
\newcommand{\sarah}{\textsf{SARAH} }
\newcommand{\sarahs}{\textsf{SARAH}}
\newcommand{\feynarts}{\textsf{FeynArts} }
\newcommand{\feynartss}{\textsf{FeynArts}}
\newcommand{\feyncalc}{\textsf{FeynCalc} }
\newcommand{\feyncalcs}{\textsf{FeynCalc}}

\newcommand{\mathematica}{\textsf{Mathematica} }
\newcommand{\CC}{C\nolinebreak\hspace{-.05em}\raisebox{.4ex}{\tiny\bf +}\nolinebreak\hspace{-.10em}\raisebox{.4ex}{\tiny\bf +} }

\usepackage{braket}

\begin{document}


\begin{figure}[h]
\includegraphics[width=0.5\textwidth]{diagrams_V0_1.pdf}\includegraphics[width=0.5\textwidth]{diagrams_V0_2.pdf}\\
\includegraphics[width=0.5\textwidth]{diagrams_V0_3.pdf}\includegraphics[width=0.5\textwidth]{diagrams_V0_4.pdf}
\end{figure}


\begin{figure}[h]
\includegraphics[width=0.5\textwidth]{diagrams_V0_5.pdf}\includegraphics[width=0.5\textwidth]{diagrams_V0_7.pdf}
\end{figure}


\begin{figure}[h]
\includegraphics[width=0.5\textwidth]{diagrams_VC_1.pdf}\includegraphics[width=0.5\textwidth]{diagrams_VC_2.pdf}\\
\includegraphics[width=0.5\textwidth]{diagrams_VC_3.pdf}\includegraphics[width=0.5\textwidth]{diagrams_VC_4.pdf}
\end{figure}


\begin{figure}[h]
\includegraphics[width=0.5\textwidth]{diagrams_VC_5.pdf}\includegraphics[width=0.5\textwidth]{diagrams_VC_7.pdf}
\includegraphics[width=0.5\textwidth]{diagrams_VC_8.pdf}
\end{figure}



\bibliography{../../../../Papers/library}{}
\bibliographystyle{science}

\end{document}  