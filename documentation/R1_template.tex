\documentclass[pdftex,twocolumn,epjc3]{svjour3}

\usepackage{fontenc}
\usepackage{calc}
\usepackage{graphicx}
\usepackage{booktabs}
\usepackage{textcomp}
\usepackage{xspace}
\usepackage{relsize}
\usepackage{amssymb}
\usepackage{amsmath}
\usepackage{listings}
\usepackage{microtype}
\usepackage{multirow}
\usepackage{tabularx}
\usepackage{array}
\usepackage[utf8]{inputenc}
\usepackage[numbers,sort&compress]{natbib}
\usepackage[labelfont=bf,font=small]{caption}
\usepackage[skip=-2pt]{subcaption}
\usepackage[colorlinks,citecolor=blue,urlcolor=blue,linkcolor=blue]{hyperref}
\usepackage[usenames,dvipsnames]{xcolor}
\usepackage[clockwise,figuresright]{rotating}

\newcolumntype{L}{>{\raggedright\let\newline\\\arraybackslash\hspace{0pt}}X}
\newcolumntype{R}{>{\raggedleft\let\newline\\\arraybackslash\hspace{0pt}}X}
\newcolumntype{C}{>{\centering\let\newline\\\arraybackslash\hspace{0pt}}X}

\setlength{\rotFPtop}{0pt plus 1fil}
\setcounter{tocdepth}{3}

\makeatletter
% \DeclareRobustCommand{\kbd}[1]{{\texttt{#1}}}
% \DeclareRobustCommand{\code}[1]{\kbd{#1}\xspace}
% \DeclareRobustCommand{\To}{\ensuremath{\Rightarrow}\xspace}
\g@addto@macro\bfseries{\boldmath}
\makeatother

\newcommand{\subparagraph}{} %< workaround for svjour not defining subparagraph
\usepackage{titlesec}
% \titleformat*{\section}{\Large\bfseries\sffamily}
% \titleformat*{\subsection}{\large\bfseries\sffamily}
% \titleformat*{\subsubsection}{\bfseries\sffamily}
\titleformat*{\paragraph}{\bfseries}
% \titleformat*{\subparagraph}{\slshape}
% \titlespacing*{\section}{0pt}{3ex plus .2ex minus .2ex}{1ex plus .2ex}
% \titlespacing*{\subsection}{0pt}{3ex plus .4ex minus .4ex}{0.8ex plus .2ex}
% \titlespacing*{\subsubsection}{0pt}{1.5ex plus .2ex minus .2ex}{0.5ex plus .2ex}
% \titlespacing*{\paragraph}{0pt}{1ex plus .1ex minus .1ex}{0.5ex plus .1ex minus .1ex}
% \titlespacing*{\subparagraph}{0pt}{0ex plus .1ex minus .1ex}{0.5ex plus .1ex minus .1ex}

\journalname{Eur. Phys. J. C}
\bibliographystyle{JHEP_pat}
\smartqed
\sloppy

\let\underscore\_
\renewcommand{\_}{\discretionary{\underscore}{}{\underscore}}

\makeatletter
\let\orgdescriptionlabel\descriptionlabel
\renewcommand*{\descriptionlabel}[1]{%
  \let\orglabel\label
  \let\label\@gobble
  \phantomsection
  \protected@edef\@currentlabel{#1}%
  %\protected@edef\@currentlabelname{#1}
  \let\label\orglabel
  \orgdescriptionlabel{#1}%
}
\makeatother

%Journal definitions
% Bibliography and bibfile
\def\lnp{Lec.\ Notes in Physics}
          % Lecture Notes in Physics
\def\cpc{Comp.\ Phys.\ Comm.}
          % Computer Physics Communications
\def\jpg{J. Phys. G}
          % Journal of Physics G Nuclear Physics
\def\ijmpa{Int.\ J.\ Mod.\ Phys.\ A}
          % International Journal of Modern Physics A
\def\epjc{Eur.\ Phys.\ J.\ C}
          % European Physical Journal C
\def\nima{Nuc.\ Inst.\ Methods A}
          % Nuclear Instruments and Methods A
\def\nimb{Nuc.\ Inst.\ Methods B}
          % Nuclear Instruments and Methods B
\def\njp{New J.\ Phys.}
          % New Journal of Physics
\def\rmp{Rev.\ Mod.\ Phys.}
          % Reviews of Modern Physics
\def\app{Astropart.\ Phys.}
          % Astroparticle Physics
\def\aj{AJ}%
          % Astronomical Journal
\def\actaa{Acta Astron.}%
          % Acta Astronomica
\def\araa{ARA\&A}%
          % Annual Review of Astron and Astrophys
\def\arnps{Ann.~Rev.~Nucl.~\& Part.~Sci.}%
          % Annual Review of Astron and Astrophys
\def\apj{ApJ}%
          % Astrophysical Journal
\def\apjl{ApJ}%
          % Astrophysical Journal, Letters
\def\apjs{ApJS}%
          % Astrophysical Journal, Supplement
\def\ao{Appl.\ Opt.}%
          % Applied Optics
\def\apss{Ap\&SS}%
          % Astrophysics and Space Science
\def\aap{A\&A}%
          % Astronomy and Astrophysics
\def\aapr{A\&A~Rev.}%
          % Astronomy and Astrophysics Reviews
\def\aaps{A\&AS}%
          % Astronomy and Astrophysics, Supplement
\def\azh{AZh}%
          % Astronomicheskii Zhurnal
\def\pos{PoS}%
          % Proceedings of Science
\def\baas{BAAS}%
          % Bulletin of the AAS
\def\bac{Bull.\ Astr.\ Inst.\ Czechosl.}%
          % Bulletin of the Astronomical Institutes of Czechoslovakia 
\def\caa{Chinese Astron.\ Astrophys.}%
          % Chinese Astronomy and Astrophysics
\def\cjaa{Chinese J.\ Astron.\ Astrophys.}%
          % Chinese Journal of Astronomy and Astrophysics
\def\icarus{Icarus}%
          % Icarus
\def\jhep{JHEP}%
          % Journal of High Energy Physics
\def\jcap{JCAP}%
          % Journal of Cosmology and Astroparticle Physics
\def\jpsj{J.\ Phys.\ Soc.\ Japan}%
          % Journal of the Physical Society of Japan
\def\jrasc{JRASC}%
          % Journal of the RAS of Canada
\def\canjphys{Can.~J.~Phys.}
          %Canadian Journal of Physics
\def\apphys{Astropart.~Phys.}
          %Astroparticle Physics
\def\mnras{MNRAS}%
          % Monthly Notices of the RAS
\def\memras{MmRAS}%
          % Memoirs of the RAS
\def\na{New A}%
          % New Astronomy
\def\nar{New A Rev.}%
          % New Astronomy Review
\def\pasa{PASA}%
          % Publications of the Astron. Soc. of Australia
\def\pra{Phys.\ Rev.\ A}%
          % Physical Review A: General Physics
\def\prb{Phys.\ Rev.\ B}%
          % Physical Review B: Solid State
\def\prc{Phys.\ Rev.\ C}%
          % Physical Review C
\def\prd{Phys.\ Rev.\ D}%
          % Physical Review D
\def\pre{Phys.\ Rev.\ E}%
          % Physical Review E
\def\prl{Phys.\ Rev.\ Lett.}%
          % Physical Review Letters
\def\pasp{PASP}%
          % Publications of the ASP
\def\pasj{PASJ}%
          % Publications of the ASJ
\def\qjras{QJRAS}%
          % Quarterly Journal of the RAS
\def\rmxaa{Rev. Mexicana Astron. Astrofis.}%
          % Revista Mexicana de Astronomia y Astrofisica
\def\skytel{S\&T}%
          % Sky and Telescope
\def\solphys{Sol.\ Phys.}%
          % Solar Physics
\def\sovast{Soviet~Ast.}%
          % Soviet Astronomy
\def\ssr{Space~Sci.\ Rev.}%
          % Space Science Reviews
\def\zap{ZAp}%
          % Zeitschrift fuer Astrophysik
\def\nat{Nature}%
          % Nature
\def\science{Science}%
\def\sci{\science}%
          % Science
\def\iaucirc{IAU~Circ.}%
          % IAU Cirulars
\def\aplett{Astrophys.\ Lett.}%
          % Astrophysics Letters
\def\apspr{Astrophys.\ Space~Phys.\ Res.}%
          % Astrophysics Space Physics Research
\def\bain{Bull.\ Astron.\ Inst.\ Netherlands}%
          % Bulletin Astronomical Institute of the Netherlands
\def\fcp{Fund.\ Cosmic~Phys.}%
          % Fundamental Cosmic Physics
\def\gca{Geochim.\ Cosmochim.\ Acta}%
          % Geochimica Cosmochimica Acta
\def\grl{Geophys.\ Res.\ Lett.}%
          % Geophysics Research Letters
\def\jcp{J.\ Chem.\ Phys.}%
          % Journal of Chemical Physics
\def\jgr{J.\ Geophys.\ Res.}%
          % Journal of Geophysics Research
\def\jqsrt{J.\ Quant.\ Spec.\ Radiat.\ Transf.}%
          % Journal of Quantitiative Spectroscopy and Radiative Trasfer
\def\memsai{Mem.\ Soc.\ Astron.\ Italiana}%
          % Mem. Societa Astronomica Italiana
\def\nphysa{Nucl.\ Phys.\ A}%
          % Nuclear Physics A
\def\nphysb{Nucl.\ Phys.\ B}%
          % Nuclear Physics B
\def\physrep{Phys.\ Rep.}%
          % Physics Reports
\def\physscr{Phys.\ Scr}%
          % Physica Scripta
\def\planss{Planet.\ Space~Sci.}%
          % Planetary Space Science
\def\procspie{Proc.\ SPIE}%
          % Proceedings of the SPIE
\def\repprogphys{Rep.\ Prog.\ Phys.}%
          % Reports of Progress in Physics
\def\jpcrd{J. Phys. Chem. Ref. Data}% 
        %Journal of Physical and Chemical Reference Data 
\def\jphysb{J. Phys. B}% 
	%Journal of Physics B Atomic Molecular Physics
\def\jphysd{J. Phys. D}% 
	%Journal of Physics D
\def\jphysconfseries{J. Phys. Conf. Series}% 
	%Journal of Physics: Conference Series
\def\physrev{\pr}
\def\pr{Phys. Rev.}% 
	%Physical Review
\def\josa{J. Opt. Soc. Amer. (1917-1983)}% 
	%Journal of the Optical Society of America (1917-1983) 
\def\josab{J. Opt. Soc. Amer. B}% 
	%Journal of the Optical Society of America B Optical Physics
\def\pla{Phys. Lett. A}% 
	%Physics Letters A
\def\plb{Phys. Lett. B}% 
	%Physics Letters B
\def\os{Opt. Spectrosc. (Russ.)}% 
	%Optics and Spectroscopy (Russ. / USSR)
\def\jas{J. Appl. Spectrosc.}% 
	%Journal of Applied Spectroscopy (Russ. / USSR)
\def\annp{Ann. Phys.}% 
	%Annalen der Physik
\def\sa{Spectrochim. Acta}% 
	%Spectrochimica Acta
\def\prsoca{Proc. R. Soc. London Ser. A}% 
	%Proceedings of the Royal Society of London, Series A
\def\zphysa{Z. Phys. A}% 
	%Zeitschrift fur Physik A
\def\zphysb{Z. Phys. B}% 
	%Zeitschrift fur Physik B
\def\zphysc{Z. Phys. C}% 
	%Zeitschrift fur Physik C
\def\zphysd{Z. Phys. D}% 
	%Zeitschrift fur Physik D
\def\zphyse{Z. Phys. E}% 
	%Zeitschrift fur Physik E
\def\zphys{Z. Phys.}% 
	%Zeitschrift fur Physik
\def\adndt{Atom. Data Nuc. Data Tables}% 
	%Atomic Data and Nuclear Data Tables
\def\jmolspec{J. Mol. Spectrosc.}% 
	%Journal of Molecular Spectroscopy
\def\aphysb{Appl. Phys. B}% 
	%Applied Physics B: Lasers and Optics
\def\nim{Nuc. Inst. Meth.}% 
	%Nuclear Instruments and Methods
\def\jphysique{J. Phys. (Paris)}% 
	%Journal de Physique
\def\epjp{Eur.~Phys.~J.~Plus}%
        %European Physical Journal Plus
\def\epjc{Eur.~Phys.~J.~C}%
        %European Physical Journal C
\def\epl{Europhys.~Lett}%
        %Europhysics Letters
\def\njp{New J.~Phys.}
        %New Journal of Physics
\let\astap=\aap
\let\apjlett=\apjl
\let\apjsupp=\apjs
\let\applopt=\ao
%


\newcommand\cpp[1]{\lstinline{#1}}
\newcommand\yaml[1]{\lstset{style=yaml}\lstinline{#1}\lstset{style=cpp}}
\newcommand\yamlvalue[1]{{\YAMLvaluestyle\ttfamily#1}}
\newcommand\term[1]{\lstset{style=terminal}\lstinline{#1}\lstset{style=cpp}}
\newcommand\fortran[1]{\lstset{style=fortran}\lstinline{#1}\lstset{style=cpp}}
\newcommand\py[1]{\lstset{style=python}\lstinline{#1}\lstset{style=cpp}}

% Ben: Placeholder command for referring to glossary items in core paper from other paper
\newcommand\corejargon[1]{\textbf{#1}}

\lstnewenvironment{lstlistingyaml}{\lstset{style=yaml}}{\lstset{style=cpp}}
\lstnewenvironment{lstlistingterm}{\lstset{style=terminal}}{\lstset{style=cpp}}
\lstnewenvironment{lstlistingfortran}{\lstset{style=fortran}}{\lstset{style=cpp}}
\lstnewenvironment{lstcpp}{\lstset{style=cpp}}{\lstset{style=cpp}}
\lstnewenvironment{lstcppalt}{\lstset{style=cppalt}}{\lstset{style=cpp}}
\lstnewenvironment{lstcppnum}{\lstset{style=cppnum}}{\lstset{style=cpp}}
\lstnewenvironment{lstyaml}{\lstset{style=yaml}}{\lstset{style=cpp}}
\lstnewenvironment{lstterm}{\lstset{style=terminal}}{\lstset{style=cpp}}
\lstnewenvironment{lsttermalt}{\lstset{style=terminalalt}}{\lstset{style=cpp}}
\lstnewenvironment{lstfortran}{\lstset{style=fortran}}{\lstset{style=cpp}}
\lstnewenvironment{lstpy}{\lstset{style=python}}{\lstset{style=cpp}}

%C++ syntax highlighting, direct from http://marcusmo.co.uk/blog/latex-syntax-highlighting/
% Solarized colour scheme for listings
\definecolor{solarized@base03}{HTML}{002B36}
\definecolor{solarized@base02}{HTML}{073642}
\definecolor{solarized@base01}{HTML}{586e75}
\definecolor{solarized@base00}{HTML}{657b83}
\definecolor{solarized@base0}{HTML}{839496}
\definecolor{solarized@base1}{HTML}{93a1a1}
\definecolor{solarized@base2}{HTML}{EEE8D5}
\definecolor{solarized@base3}{HTML}{FDF6E3}
\definecolor{solarized@yellow}{HTML}{B58900}
\definecolor{solarized@orange}{HTML}{CB4B16}
\definecolor{solarized@red}{HTML}{DC322F}
\definecolor{solarized@magenta}{HTML}{D33682}
\definecolor{solarized@violet}{HTML}{6C71C4}
\definecolor{solarized@blue}{HTML}{268BD2}
\definecolor{solarized@cyan}{HTML}{2AA198}
\definecolor{solarized@green}{HTML}{859900}
\definecolor{darkred}{HTML}{550003}
\newcommand\YAMLcolonstyle{\footnotesize\color{solarized@red}\mdseries}
\newcommand\YAMLstringstyle{\footnotesize\color{solarized@green}\mdseries}
\newcommand\YAMLkeystyle{\footnotesize\color{solarized@blue}\ttfamily}
\newcommand\YAMLvaluestyle{\footnotesize\color{blue}\mdseries}
\newcommand\ProcessThreeDashes{\llap{\color{cyan}\mdseries-{-}-}}
% Define C++ syntax highlighting colour scheme
\newcommand\CPPplainstyle{\footnotesize\ttfamily}
\newcommand\CPPkeywordstyle{\color{solarized@orange}\footnotesize\ttfamily}
\newcommand\CPPidentifierstyle{\color{solarized@blue}\footnotesize\ttfamily}
\lstdefinestyle{cpp}
{
  language=C++,
  basicstyle=\footnotesize\ttfamily,
  basewidth={0.53em,0.44em}, %Ben: experimenting a bit with the fixed-width width (first argument); feels a bit more readable to me with the slightly smaller width (was 0.6em by default)
  numbers=none,
  tabsize=2,
  breaklines=true,
  escapeinside={@}{@},
  showstringspaces=false,
  numberstyle=\tiny\color{solarized@base01},
  keywordstyle=\color{solarized@orange},
  stringstyle=\color{solarized@red}\ttfamily,
  identifierstyle=\color{solarized@blue},
  commentstyle=\color{solarized@violet},
  morecomment=[l]{\#pragma},
  emphstyle=\color{solarized@green},
  frame=single,
  rulecolor=\color{solarized@base2},
  rulesepcolor=\color{solarized@base2},
  otherkeywords={\#define,\#undef,\#include},
  literate =
}
% C++ style with different escape character (so I can use @'s in strings)
\lstdefinestyle{cppalt}
{
  language=C++,
  basicstyle=\footnotesize\ttfamily,
  basewidth={0.53em,0.44em}, %Ben: experimenting a bit with the fixed-width width (first argument); feels a bit more readable to me with the slightly smaller width (was 0.6em by default)
  numbers=none,
  tabsize=2,
  breaklines=true,
  escapeinside={*@}{@*},
  showstringspaces=false,
  numberstyle=\tiny\color{solarized@base01},
  keywordstyle=\color{solarized@orange},
  stringstyle=\color{solarized@red}\ttfamily,
  identifierstyle=\color{solarized@blue},
  commentstyle=\color{solarized@violet},
  emphstyle=\color{solarized@green},
  frame=single,
  rulecolor=\color{solarized@base2},
  rulesepcolor=\color{solarized@base2},
  otherkeywords={define,undef,include,\#},
  literate =
}
% C++ style with line numbers (try to keep everything else matching the 'cpp' style)
\lstdefinestyle{cppnum}
{
  language=C++,
  basicstyle=\footnotesize\ttfamily,
  basewidth={0.53em,0.44em}, %Ben: experimenting a bit with the fixed-width width (first argument); feels a bit more readable to me with the slightly smaller width (was 0.6em by default)
  numbers=left,
  tabsize=2,
  breaklines=true,
  escapeinside={@}{@},
  showstringspaces=false,
  numberstyle=\tiny\color{solarized@base01},
  keywordstyle=\color{solarized@orange},
  stringstyle=\color{solarized@red}\ttfamily,
  identifierstyle=\color{solarized@blue},
  commentstyle=\color{solarized@violet},
  emphstyle=\color{solarized@green},
  frame=single,
  rulecolor=\color{solarized@base2},
  rulesepcolor=\color{solarized@base2},
  otherkeywords={define,undef,include,\#},
  literate =
}
% Define python syntax highlighting colour scheme
\lstdefinestyle{python}
{
  language=Python,
  basicstyle=\footnotesize\ttfamily,
  basewidth={0.53em,0.44em},
  numbers=none,
  tabsize=2,
  breaklines=true,
  escapeinside={@}{@},
  showstringspaces=false,
  numberstyle=\tiny\color{solarized@base01},
  keywordstyle=\color{blue},
  stringstyle=\color{orange}\ttfamily,
  identifierstyle=\color{darkred},
  commentstyle=\color{purple},
  emphstyle=\color{green},
  frame=single,
  rulecolor=\color{solarized@base2},
  rulesepcolor=\color{solarized@base2},
  literate = {\ as\ }{{\color{blue}\ as\ }}3 
}
% Define fortran syntax highlighting colour scheme
\lstdefinestyle{fortran}
{
  language=Fortran,
  basicstyle=\footnotesize\ttfamily,
  basewidth={0.53em,0.44em},
  numbers=none,
  tabsize=2,
  breaklines=true,
  escapeinside={@}{@},
  showstringspaces=false,
  numberstyle=\tiny\color{solarized@base01},
  keywordstyle=\color{blue},
  stringstyle=\color{orange}\ttfamily,
  identifierstyle=\color{black},
  commentstyle=\color{purple},
  emphstyle=\color{green},
  frame=single,
  rulecolor=\color{solarized@base2},
  rulesepcolor=\color{solarized@base2},
}
% Define shell syntax highlighting colour scheme
% Ben: I cannot get the damn comment highlighting to work for the 'bash' language. No idea what the problem is, the internet seems to think that it should just work.
\lstdefinestyle{terminal}
{
  language=bash,
  basicstyle=\footnotesize\ttfamily,
  numbers=none,
  tabsize=2,
  breaklines=true,
  escapeinside={@}{@},
  frame=single,
  showstringspaces=false,
  numberstyle=\tiny\color{solarized@base01},
  keywordstyle=\color{solarized@orange},
  stringstyle=\color{solarized@red}\ttfamily,
  identifierstyle=\color{black},
  commentstyle=\color{solarized@violet},
  emphstyle=\color{solarized@green},
  frame=single,
  rulecolor=\color{solarized@base2},
  rulesepcolor=\color{solarized@base2},
  morekeywords={gambit, cmake, make, mkdir},
  deletekeywords={test},
  literate = {\ gambit}{{\ }{\color{black}}gambit}7
             {/gambit}{{/}{\color{black}}gambit}6
             {/include}{{/}{\color{black}}include}8
             {cmake/}{{\color{black}}cmake/}6
             {.cmake}{{.}{\color{black}}cmake}6
}
% Terminal style with alternate escape character
\lstdefinestyle{terminalalt}
{
  language=bash,
  basicstyle=\footnotesize\ttfamily,
  numbers=none,
  tabsize=2,
  breaklines=true,
  escapeinside={*@}{@*},
  frame=single,
  showstringspaces=false,
  numberstyle=\tiny\color{solarized@base01},
  keywordstyle=\color{solarized@orange},
  stringstyle=\color{solarized@red}\ttfamily,
  identifierstyle=\color{black},
  commentstyle=\color{solarized@violet},
  emphstyle=\color{solarized@green},
  frame=single,
  rulecolor=\color{solarized@base2},
  rulesepcolor=\color{solarized@base2},
  morekeywords={gambit, cmake, make, mkdir},
  deletekeywords={test},
  literate = {\ gambit}{{\ }{\color{black}}gambit}7
             {/gambit}{{/}{\color{black}}gambit}6
             {/include}{{/}{\color{black}}include}8
             {cmake/}{{\color{black}}cmake/}6
             {.cmake}{{.}{\color{black}}cmake}6
}
\newcommand{\negphantom}[1]{\settowidth{\dimen0}{#1}\hspace*{-\dimen0}}
% Define yaml syntax highlighting colour scheme
\lstdefinestyle{yaml}
{
  escapeinside={@}{@},
  keywords={true,false,null},
  otherkeywords={},
  keywordstyle=\color{solarized@base0}\bfseries,
  basicstyle=\footnotesize\color{black}\ttfamily,
  identifierstyle=\YAMLkeystyle,
  sensitive=false,
  commentstyle=\color{solarized@orange}\ttfamily,
  morecomment=[l]{\#},
  morecomment=[s]{/*}{*/},
  stringstyle=\YAMLstringstyle\ttfamily,
  moredelim=**[s][\YAMLkeystyle]{,}{:},   % switch to value style at : but back to key style at,
  moredelim=**[l][\YAMLvaluestyle]{:},    % switch to value style at :
  morestring=[b]',
  morestring=[b]",
  literate =    {---}{{\ProcessThreeDashes}}3
                {>}{{\textcolor{solarized@red}\textgreater}}1
                {|}{{\textcolor{solarized@red}\textbar}}1
                {\ -\ }{{\mdseries\color{black}\ -\ \negmedspace}}3
                {\}}{{{\color{black} \}}}}1
                {\{}{{{\color{black} \{}}}1
                {[}{{{\color{black} [}}}1
                {]}{{{\color{black} ]}}}1,
  breakindent=0pt,
  breakatwhitespace,
  columns=fullflexible
}
% Start with C++ style on
\lstset{style=cpp}

% Glossary commands
\newcommand{\cross}[1]{\ref{#1}}
\newcommand{\doublecross}[2]{\hyperref[#2]{\textbf{#1}}}
\newcommand{\doublecrosssf}[2]{\hyperref[#2]{\textbf{\textsf{#1}}}}
\newcommand{\gitem}[1]{\item[\textbf{#1}\label{#1}]}
\newcommand{\gsfitem}[1]{\item[\textbf{\textsf{#1}}\label{#1}]}

% Code commands
\newcommand{\bcode}{\begin{lstlisting}}
\newcommand{\ecode}{\end{lstlisting}}
\newcommand{\metavar}[1]{\textit{\color{black!70}\footnotesize\textrm{#1}}}

% For sign(mu), etc.
\DeclareMathOperator{\sign}{sign}

% Physics units
\newcommand{\eV}{\ensuremath{\text{e}\mspace{-0.8mu}\text{V}}\xspace}
\newcommand{\MeV}{\text{M\eV}\xspace}
\newcommand{\GeV}{\text{G\eV}\xspace}
\newcommand{\TeV}{\text{T\eV}\xspace}
\newcommand{\pb}{\text{pb}\xspace}
\newcommand{\fb}{\text{fb}\xspace}
\newcommand{\invpb}{\ensuremath{\pb^{-1}}\xspace}
\newcommand{\invfb}{\ensuremath{\fb^{-1}}\xspace}

% Physical quantities
\newcommand{\pt}{\ensuremath{p_\mathrm{T}}\xspace}
\newcommand{\et}{\ensuremath{E_\mathrm{T}}\xspace}
\newcommand{\etmiss}{\ensuremath{E_\mathrm{T}^\mathrm{\mspace{1.5mu}miss}}\xspace}
\newcommand{\hT}{\ensuremath{H_\mathrm{T}}\xspace}
\newcommand{\dphi}{\ensuremath{\Delta\phi}\xspace}
\newcommand{\lhs}{\lambda_{h\sss S}}
\newcommand{\ls}{\lambda_{\sss S}}
\newcommand{\DR}{$\overline{DR}$\xspace}
\newcommand{\MSbar}{$\overline{MS}$\xspace}

% Textual shortcuts
\newcommand{\ie}{\textit{i.e.}\ }
\newcommand{\eg}{\textit{e.g.}\ }
\newcommand{\atlas}{ATLAS\xspace}
\newcommand{\cms}{CMS\xspace}
\newcommand{\gambit}{\textsf{GAMBIT}\xspace}
\newcommand{\darkbit}{\textsf{DarkBit}\xspace}
\newcommand{\colliderbit}{\textsf{ColliderBit}\xspace}
\newcommand{\flavbit}{\textsf{FlavBit}\xspace}
\newcommand{\specbit}{\textsf{SpecBit}\xspace}
\newcommand{\decaybit}{\textsf{DecayBit}\xspace}
\newcommand{\precisionbit}{\textsf{PrecisionBit}\xspace}
\newcommand{\scannerbit}{\textsf{ScannerBit}\xspace}
\newcommand{\examplebita}{\textsf{ExampleBit\_A}\xspace}
\newcommand{\examplebitb}{\textsf{ExampleBit\_B}\xspace}
\newcommand{\BOSS}{\textsf{BOSS}\xspace}
\newcommand{\GB}{\gambit}
\newcommand{\DB}{\darkbit}
\newcommand{\omp}{\textsf{OpenMP}\xspace}
\newcommand{\openmpi}{\textsf{OpenMPI}\xspace}
\newcommand{\mpi}{\textsf{MPI}\xspace}
\newcommand{\posix}{\textsf{POSIX}\xspace}
\newcommand{\buckfast}{\textsf{BuckFast}\xspace}
\newcommand{\delphes}{\textsf{Delphes}\xspace}
\newcommand{\pythia}{\textsf{Pythia}\xspace}
\newcommand{\pythiaeight}{\textsf{Pythia 8}\xspace}
\newcommand{\PythiaEM}{\textsf{PythiaEM}\xspace}
\newcommand{\prospino}{\textsf{Prospino}\xspace}
\newcommand{\nllfast}{\textsf{NLL-fast}\xspace}
\newcommand{\madgraph}{\textsf{MadGraph}\xspace}
\newcommand{\fastjet}{\textsf{FastJet}\xspace}
\newcommand{\smodels}{\textsf{SmodelS}\xspace}
\newcommand{\fastlim}{\textsf{Fastlim}\xspace}
\newcommand{\checkmate}{\textsf{CheckMATE}\xspace}
\newcommand{\higgsbounds}{\textsf{HiggsBounds}\xspace}
\newcommand{\susypope}{\textsf{SUSYPope}\xspace}
\newcommand{\higgssignals}{\textsf{HiggsSignals}\xspace}
\newcommand{\ds}{\textsf{DarkSUSY}\xspace}
\newcommand{\pppc}{\textsf{PPPC4DMID}\xspace}
\newcommand{\micromegas}{\textsf{micrOMEGAs}\xspace}
\newcommand{\rivet}{\textsf{Rivet}\xspace}
\newcommand{\feynrules}{\textsf{Feynrules}\xspace}
\newcommand{\feynhiggs}{\textsf{FeynHiggs}\xspace}
\newcommand\SARAH{\textsf{SARAH}\xspace}
\newcommand\SPheno{\textsf{SPheno}\xspace}
\newcommand\superiso{\textsf{SuperIso}\xspace}
\newcommand\FlexibleSUSY{\textsf{FlexibleSUSY}\xspace}
\newcommand\SOFTSUSY{\textsf{SOFTSUSY}\xspace}
\newcommand\SUSPECT{\textsf{SUSPECT}\xspace}
\newcommand\NMSSMCalc{\textsf{NMSSMCALC}\xspace}
\newcommand\NMSSMTools{\textsf{NMSSMTools}\xspace}
\newcommand\NMSPEC{\textsf{NMSPEC}\xspace}
\newcommand\NMHDECAY{\textsf{NMHDECAY}\xspace}
\newcommand\HDECAY{\textsf{HDECAY}\xspace}
\newcommand\SDECAY{\textsf{SDECAY}\xspace}
\newcommand\SUSYHIT{\textsf{SUSYHIT}\xspace}
\newcommand\SFOLD{\textsf{SFOLD}\xspace}
\newcommand\HFOLD{\textsf{HFOLD}\xspace}
\newcommand\FeynHiggs{\textsf{FeynHiggs}\xspace}
\newcommand\Mathematica{\textsf{Mathematica}\xspace}
\newcommand\lilith{\textsf{Lilith}\xspace}
\newcommand\nulike{\textsf{nulike}\xspace}
\newcommand\gamLike{\textsf{gamLike}\xspace}
\newcommand\pippi{\textsf{pippi}\xspace}
\newcommand\MultiNest{\textsf{MultiNest}\xspace}
\newcommand\gamlike{\textsf{GamLike}\xspace}
\newcommand\ddcalc{\textsf{DDCalc}\xspace}
\newcommand\xx{\raisebox{0.2ex}{\smaller ++}\xspace}
\newcommand\Cpp{\textsf{C\xx}\xspace}
\newcommand\Cppeleven{\textsf{C\raisebox{0.2ex}{\smaller ++}11}\xspace}
\newcommand\plainC{\textsf{C}\xspace}
\newcommand\Python{\textsf{Python}\xspace}
\newcommand\python{\Python}
\newcommand\Fortran{\textsf{Fortran}\xspace}

\newcommand{\beq}{\equation}
\newcommand{\eeq}{\endequation}
\newcommand{\sss}{\scriptscriptstyle}
\newcommand{\ms}{m_{\sss S}}
\newcommand{\mail}[1]{\href{mailto:#1}{#1}}

% Author comments
\newcommand{\TODO}[1]{\textbf{\textcolor{red}{#1}}}
\newcommand{\tb}[1]{{\color{green}\textbf{[TB: #1]}}}
\newcommand{\ps}[1]{\Pat{: #1}}
\newcommand{\cw}[1]{{\color{red}Christoph: #1}}
\newcommand{\gm}[1]{{\color{violet}Greg: #1}}
\newcommand{\Ben}[1]{{\bf\color{magenta}Ben #1}}
\newcommand{\Csaba}[1]{{\bf\color{orange}Csaba #1}}
\newcommand{\Chris}[1]{{\bf\color{olive}Chris #1}}
\newcommand{\Pat}[1]{{\bf\color{blue}Pat #1}}
\newcommand{\Peter}[1]{{\bf\color{purple}Peter #1}}
\newcommand{\Anders}[1]{{\bf\color{brown}Anders #1}}
\newcommand{\James}[1]{{\bf\color{teal}James #1}}
\newcommand{\Abram}[1]{{\bf\color{black!20!gray!60!magenta}Abram: #1}}

% Custom \chapter-like command  (svjour3 document class does not define \part or \chapter)
\newcommand{\segment}[1]{
 {\clearpage\noindent\phantomsection\huge\it#1\par}
 {\addcontentsline{toc}{section}{\it#1}}
}
